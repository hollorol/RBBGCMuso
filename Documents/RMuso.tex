\documentclass[12pt,a4paper]{article}
\usepackage[utf8]{inputenc}
\usepackage[magyar]{babel}
\usepackage[T1]{fontenc}
\usepackage{amsmath}
\usepackage{amsfonts}
\usepackage{amssymb}
\usepackage{graphicx}
\usepackage[left=2cm,right=2cm,top=2cm,bottom=2cm]{geometry}
\frenchspacing

\title{Az RMuso csomag használata}
%\author{Hollós Roland}


\begin{document}
\maketitle
\section{A csomag alapfüggvényei}

Az R-muso csomag az alábbi fő függvényeket használja:

\begin{itemize}
\item muso.R
\item changecontent.R
\item setup.R
\item other\_usefull\_functions.R
\item getOutput.R
\end{itemize}
\section{Tartalmazott függvény-csoportok áttekintése}
\subsubsection*{setup.R}
Ebben az R fileban találhatjuk a \textit{setup()} függvényt, amelyben beállíthatjuk a fő paramétereket, mint például a végrehajtható fájl(muso.exe) neve/elérési útvonala, az outputfile-ok elérési helye.
\subsubsection*{muso.R}
Itt olyan fügvények találhatók, mint a \textit{rungetMuso()}, valamint a \textit{spinupMuso()} és a \textit{normalMuso()}. Az előbbi bekéri a bemeneti paramétereket.  
\subsubsection*{chagecontent.R}
Itt definiáltuk azt a függvényt( \textit{changemulline()}), ami alkalmas arra, hogy adott fájlnak, adott sorait, adott tartalmakra cserélje.
\subsubsection*{other\_usefull\_functions.R}
Két függvényt tartalmaz: \textit{getyearlycum()} és \textit{getyearlymax()} Az előbbi éves összegzést végez a megfelelő változóra vonatkozóan, az utóbbi pedig meghatározza az éves maximumértéket.
\subsubsection*{getOutput.R}
Ez a file felelős az output adadok beolvasásáért, a \textit{getdailyout()}, a  \textit{getmonthlyout()} és a  \textit{getyearlyout()} függvények nevüknek megfelelő formátum szerint adják vissza a kimeneti értékeket.
\section{Használat}

\subsubsection*{setup()}
A függvény, mely beállítja a környezetet a MuSo számára összesen 8 paramétert fogad el, ezek sorban: executable, parallel, calibrationpar, outputloc, inputloc, metinput, ininput, epcinput. Ezekből 11 elemű lista képződik, amit a rungetMuso igényel. Részletesebben:

\begin{itemize}
\item \textbf{executable}: A muso/muso.exe file helye és neve.
\item \textbf{calibrationpar}: A kalibráláshoz kiválasztott változók sorszáma az epc file-ban.
\item \textbf{inputloc}: Az .ini file-ok helye.
\item \textbf{metinput}: A met. file-ok helye.
\item \textbf{epcinput}: Az epc file-ok helye.
\end{itemize}

Alapértelmezés szerint a végrehatjható fájl az ini, a met, és az epc fájlokkal megegyező helyen található.\\  
\newline \newline
\textbf{Példa:}\\
  settings=setup(executable = "executable", calibrationpar = c(,,,), inputloc = "inputloc", metimput = "metinput", ininput = "ininput", epcinput = "epcinput")
\subsubsection*{rungetMuso}
A rungetMuso az RMuso fő függvénye. 3 dolgot csinál: a felhasználó igényei alapján megváltoztatja a bemeneti paramétereket, futtatja spinup és normal módban a MuSo-t, kiírja a kimeneti fájlokat igény szerint. Elsődlegesen a képernyőre, de az könnyen átirányítható egy változóba. \par

Mindehhez az alábbi 3 bemenetre van szüksége:

\begin{itemize}
\item \textbf{settings}: ez egy 11 elemű lista, amit akár a setup() függvénnyel is legenerálhatunk.
\item \textbf{parameters}: ez a változó tartalmazza az új értékeket.
\item \textbf{timee}: A változó tartalmazza, hogy az output milyen felbontásban érkezzen. 3 értéke lehet: "d"(napi) ,"m" (havi),"y" (évi). Az évi az alapértelmezett.
\end{itemize}
\newline \newline
\textbf{Példa:}\newline \newline
I. mód:\newline
settings<-setup(...)\\
rungetMuso(settings,parameters,timee="y")\\
II.mód:\newline
rungetMuso(setup(..),parameters,timee="y")

\end{document}
